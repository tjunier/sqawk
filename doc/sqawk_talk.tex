\startmode[presentation]

\setuppapersize[S6][S6]

\setuplayout [width=fit,
	rightmargin=1.5cm,
	leftmargin=1.5cm,
	leftmargindistance=0pt,
	rightmargindistance=0pt,
	height=fit,
	header=0pt,
	footer=5pt,
	topspace=1.5cm,
	backspace=1.5cm,
	bottomspace=.8cm,
	bottom=12pt,
	location=singlesided]

% Fonts

\usetypescript[palatino] [ec]
\setupbodyfont[palatino,24pt] % or maybe 20pt, Palatino is big!

\setupcolors[state=start]
\setuppagenumber[state=stop]

\setupbackgrounds[page][background=color,backgroundcolor=darkblue]
\startcolor[white]

\setupitemize[each][packed][]

\definestartstop[Slide]
	[commands={\switchtobodyfont[24pt]},
	before=\page]

\define[1]\SlideTitle
	{\midaligned{\switchtobodyfont[palatino,30pt]{\bf #1}}
	\blank[line]}

\setupinteraction[page=yes,
	color=InteractionColor,
	contrastcolor=ContrastColor,
	menu=on,
	state=start]

\startinteractionmenu[bottom]
\rightaligned{\interactionbar[alternative=f,width=\makeupwidth,height=1ex]}
\stopinteractionmenu

\setuptolerance[verytolerant,stretch]

\stopmode

\startmode[manuscript]
\setupframedtexts[location=middle,
	before={\blank[line]},
	after={\blank[line]}]
\let\startSlide\startframedtext
\let\stopSlide\stopframedtext

\define[1]\SlideTitle{\midaligned{#1}\blank[small]}

\stopmode

\defineframedtext[Code][
	foregroundstyle={\tt},
	background=color,
	backgroundcolor=lightgray,
	foregroundcolor=black,
	strut=yes,
	frame=off,
	corner=round,
	width=local,
	align=right,
	location=middle,
	offset=0.3ex]

\define[1]\Shell{\Code{\$ #1}}

\define[1]\Diff{\color[orange]{#1}}

\definedescription[OutThere][
	location=serried,headstyle=bold,width=broad]

\usesymbols[mvs]

\hyphenation{SELECT}

\define \sqawk{sqawk}
\define \cli{\cap{cli}}
\define \csv{\cap{csv}}
\define \sql{\cap{sql}} 


\starttext

\startSlide
\SlideTitle{\sqawk}
\startalignment[middle]
or \par
\cli{} \sql{} for \csv :-) \par
i.e. \par
making some shell tasks on \csv{} files easier
\stopalignment
\stopSlide

\startSlide
\SlideTitle{Comma-separated Values}
\vfill
\rightarrow{} {\em lingua franca} of tabular data (e.g., relational databases)
\vfill
\stopSlide

\startSlide
\SlideTitle{Simple \csv{} file tasks}
\vfill
\csv{} is text \Rightarrow{} work in the shell
\vfill
\startalignment[lohi]
\startitemize[a,unpacked]
	\item extract all rows of {\tt myfile} where the value in field \#3 is greater than 12.5
	\item join {\tt file1} with {\tt file2} on a common field 
\stopitemize
\stopalignment
\vfill
\stopSlide

\startSlide
\SlideTitle{Shell one-liners}
\vfill
\startitemize[a,unpacked]
	\item \Shell{awk '{\$3 > 12.5}' < myfile}
	\item \Shell{join file1 file2} \par
	{\small supposing both files are sorted on the join field, in this case \#1}
\stopitemize
\vfill
\stopSlide

\startSlide
\SlideTitle{Similar tasks} 
\startitemize[a,unpacked][stopper={'.}]
	\item extract all rows of {\tt myfile} where the value in column 3 \Diff{is above average}
	\item join {\tt file1} with {\tt file2}, but \Diff{on a composite join field} (e.g. hospital ID + patient ID)
	\item join \Diff{more than two files} (e.g. genotypes, covariates, eigenvalues)
\stopitemize
\stopSlide

\startSlide
\SlideTitle{{\em No} one-liners!}
\startitemize[a,unpacked][stopper={'.}]
	\item {\tt awk} must have read all of column 3 to compute average
	\item {\tt join} needs both files sorted (extra steps, destroys order); can't use $> 1$ field 
	\item {\tt join} cannot handle $> 2$ files
\stopitemize
\stopSlide

\startSlide
\SlideTitle{Personal postulate}
\vfill
There is a category of tasks that can {\em almost} be done with one-liners,
      but not quite
\vfill
This is a shame \symbol[martinvogel 2][Smiley]
\vfill
And as a matter of fact\ldots
\vfill
\stopSlide

\startSlide
\SlideTitle{As Database Operations}
\ldots{} you can express them as a single \sql{} query:
\startitemize[a,packed][stopper={'.}]
	\item \Code{SELECT * FROM t1 WHERE f3 > (SELECT avg(f3));}
	\item \Code{SELECT t1.* FROM t1 JOIN t2 USING (f1,f2,f3);}
	\item \Code{SELECT ... FROM t1 JOIN t2 ... JOIN t3 ...;}
\stopitemize
\switchtobodyfont[14.4pt]{where tables ({\tt t1}, etc.) contain files, columns ({\tt f1}, etc.) contain file fields}
\stopSlide

\startSlide
\SlideTitle{but\ldots}
to do \sql{} queries on a file, you need:
\startitemize
	\item to create a database
	\item to create a table for the file
	\item to import the file's data into the table
	\item (usually) a \sql{} server 
\stopitemize
\ldots which is a bit unwieldy.
\stopSlide

\startSlide
\SlideTitle{in other words}
\midaligned{
\starttable[|r|l|l|]
\NC \NC {\bf pros} \NC {\bf cons} \SR
\HL
\NC shell 1-liners \NC concise \NC limited \NC\AR
\NC \sql{} \NC expressive \NC awkward \SR
\stoptable
}
\startproperty[hidden]
$\rightarrow$ Is it possible to have the concision and the expressiveness?
\stopproperty
\stopSlide

\startSlide
\SlideTitle{in other words}
\midaligned{
\starttable[|r|l|l|]
\NC \NC {\bf pros} \NC {\bf cons} \SR
\HL
\NC shell 1-liners \NC concise \NC limited \NC\AR
\NC \sql{} \NC expressive \NC awkward \SR
\stoptable
}
$\rightarrow$ Can we have both the concision and the expressiveness?
\stopSlide

\startSlide
\SlideTitle{Wish List}
the successful candidate will:
\startitemize
	\item automatically create a database
	\item automatically create db tables from \csv{} files
	\item automatically import content into the tables
	\item run a \sql{} query
	\item print out the result
	\item be a shell filter
\stopitemize
\stopSlide

\startSlide
\SlideTitle{Anything Out There?}
\startOutThere{\sql{}-Powered Awk}
Functions for accessing My\sql{}  databases from Awk
\stopOutThere
\startOutThere{ShellSQL} Programs that enable shell scripts to connect to \sql{} engines
\stopOutThere
\vfill
\midaligned{$\rightarrow$ not exactly what I want}
\stopSlide

\startSlide
\SlideTitle{SQLite}
\startitemize
\item C library (not server)
\item small, fast
\stopitemize
$\rightarrow$ if we can automate table creation and population, we're done.
\stopSlide

\startSlide
\SlideTitle{Results}
\stopSlide

\startSlide
\SlideTitle{\sqawk}
{\small (or "squawk", need a better name anyway\ldots clisql?)}
\startitemize[n,repeat]
	\item creates a database (memory -> transient)
	\item for each \csv{} file (or stdin):
	\startitemize[n]
		\item examines first few lines
		\item creates a table with appropriate names and types
	\stopitemize
	\item runs the \sql{} query
	\item prints out the result rows
\stopitemize
\stopSlide

\startSlide
\SlideTitle{Syntax}
\Shell {\sqawk{} [opts] ([file opts] file)+ SQL}
\vfill
Selected options:
\startitemize[unpacked]
	\item {\tt\bf -i}: specifies index fields
	\item {\tt\bf -p}: specifies primary key
	\item {\tt\bf -q}: shows the generated SQL
\stopitemize
\stopSlide

\startSlide
\SlideTitle{Examples}
\startitemize[a][stopper={'.}]
	\item \Shell{sqawk myfile.csv 'SELECT * FROM myfile WHERE f3 > (SELECT avg(f3))'}
	\item \Code{sqawk file1.csv file2.csv 'SELECT * FROM file1 JOIN file2 USING (f1,f2,f3);}
\stopitemize
\stopSlide

\startSlide
\SlideTitle{Checking for valid IDs}
\vfill
\startitemize
\item file \filename{valid}: list of valid IDs
\item file \filename{dubious}: uncertain IDs (among other data)
\stopitemize
\vfill
\Shell{./sqawk valid dubious 'SELECT * FROM dubious WHERE dubious.spc NOT IN (SELECT spc FROM valid)'}
\vfill
\stopSlide

% \startSlide
% \SlideTitle{Fixing field order}
% Received Excel tables with different ordering of fields
% \vfill
% \Shell{\typefile{../example_3}}
% \vfill
% \stopSlide
% 
% \startSlide
% \SlideTitle{Intersecting VCFs}
% \vfill
% \Shell{\typefile{../example_2}}
% \vfill
% \stopSlide

\startSlide
\SlideTitle{Conclusion}
\startitemize
	\item \sqawk{} makes working with \csv{} and \csv{}-like files easier 
	tasks
\stopitemize
\stopSlide

\startSlide
\SlideTitle{In an ideal world:}
\startitemize
\item data is consistently formatted
\item data formats are compatible
\item data is validated before use
\item \ldots
\stopitemize
\stopSlide

\startSlide
\SlideTitle{In the real world\ldots}
\startitemize
\item data is messy
\item there is a plethora of incompatible formats
\item nobody checks the data before sending it :-)
\item \ldots
\stopitemize
\stopSlide

\startSlide
\SlideTitle{what can be done?}
\startitemize
\item export to \csv{}
\item write code to systematically check the data
\item \ldots
\stopitemize
\Rightarrow{} this is where \sqawk{} might help.
\stopSlide

\startSlide
\SlideTitle{That's all}
\vfill
Thanks
\vfill
\stopSlide



\stoptext
